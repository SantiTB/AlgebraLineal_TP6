\documentclass{article}
\usepackage[margin=0.5in]{geometry}
\usepackage{amsmath}
\usepackage{graphicx}
\usepackage{amssymb}
\usepackage{multicol}
\usepackage{xcolor}
\usepackage{amsthm}
\usepackage{mdframed}

\newenvironment{tightcenter}{%
    \setlength\topsep{0pt}
    \setlength\parskip{0pt}
    \begin{center}
}{%  
    \end{center}
}
\newcommand{\R}{\mathbb{R}}
\newcommand{\Z}{\mathbb{Z}}
\newcommand{\C}{\mathbb{C}}
\newcommand{\K}{\mathbb{K}}
\newcommand{\PP}{\mathbb{P}}
\newcommand{\E}{\mathcal{E}}
\title{Trabajo Práctico 6 - Formas canónicas elementales II}
\author{Santiago}
\date{}
\begin{document}
    \maketitle
    \global\mdfdefinestyle{s}{%
            linecolor=orange,linewidth=0.5pt,%
            leftmargin=0cm,rightmargin=1cm
        }
    \begin{enumerate}
        \item Sea $A$ una matriz tal que $A^2=A$ pero $A\neq I,0$.
    \begin{enumerate}
        \item Hallar el polinomio minimal de $A$.
            \begin{mdframed}[style=s]
                
            \end{mdframed}
        \item Probar que $A$ es semejante a la matriz diagonal $\begin{pmatrix}
                I_r&0\\0&0
            \end{pmatrix}$, donde $r=Rg(A)$.
            \begin{mdframed}[style=s]
                
            \end{mdframed}
    \end{enumerate}
        \item Sea $T\in L(\R^4)$ tal que\[[T]_\E=\begin{pmatrix}
        1&1&0&0\\0&0&0&0\\0&1&0&0\\0&0&0&1
    \end{pmatrix}\]
    \begin{enumerate}
        \item Probar que los únicos autovalores de $T$ son 0 y 1, pero $T$ no es una proyección.
            \begin{mdframed}[style=s]
                
            \end{mdframed}
        \item ¿Es diagonalizable $T$?
            \begin{mdframed}[style=s]
                
            \end{mdframed}
        \item Sea $S$ un operador diagonalizable que tiene como únicos autovalores al 0 y al 1 ¿Se puede afirmar que $S$ es una proyección?
            \begin{mdframed}[style=s]
                
            \end{mdframed}
    \end{enumerate}
        \item Probar que los siguientes operadores son proyecciones:
    \begin{enumerate}
        \item $T:\R_3[x]\to\R_3[x]$ dado por $T(ax^2+bx+c)=c$.
            \begin{mdframed}[style=s]
                
            \end{mdframed}
        \item $diag:\C^{n\times n}\to\C^{n\times n}$ dado por\[(diag(A))_{ij}=\begin{cases}
                A_{ii}&\quad\text{si }i=j\\
                0&\quad\text{si }i\neq j
            \end{cases}\qquad\text{para }i,j=1,\dots,n.\]
            \begin{mdframed}[style=s]
                
            \end{mdframed}
    \end{enumerate}
        \item Sea \[A=\begin{pmatrix}
        2&-1&0\\-1&2&-1\\0&-1&2
    \end{pmatrix}\]
    (la matriz del ejercicio 1 del TP 5) y supongamos que $A=[T]_\E$ para $T\in L(\C^3)$. Hallar tres subespacios no nulos de $\C^3$ que sean invariantes por $T$ y tales que $\C^3$ se pueda escribir como suma directa de ellos.
    \begin{mdframed}[style=s]
        
    \end{mdframed}
        \item \begin{enumerate}
        \item Descomponer a $\R^3$ como suma directa de subespacios $W_1,W_2,W_3$.
            \begin{mdframed}[style=s]
                
            \end{mdframed}
        \item Hallar las proyecciones $P_1,P_2,P_3$ correspondientes a cada uno de los subespacios del inciso $a$,\\respectivamente.
            \begin{mdframed}[style=s]
                
            \end{mdframed}
        \item Hallar el polinomio minimal y el característico de $\sqrt{2}P_1+\pi P_2+3P_3$.
            \begin{mdframed}[style=s]
                
            \end{mdframed}
        \item ¿Es $\sqrt{2}P_1+\pi P_2+3P_3$ un operador diagonalizable? Hallar sus autovalores y autovalores correspondientes.
            \begin{mdframed}[style=s]
                
            \end{mdframed}
    \end{enumerate}
        \item Sea \[A=\begin{pmatrix}
        0&0&0&1\\0&0&1&0\\0&-1&0&1\\-1&0&0&0
    \end{pmatrix},\]
    una representación matricial de $T\in L(\C^4)$ (pensando a $\C^4$ como $\C$-EV)
    \begin{enumerate}
        \item Hallar el polinomio característico de $T$ y el minimal. ¿Es $T$ diagonalizable?
            \begin{mdframed}[style=s]
                
            \end{mdframed}
        \item Hallar dos subespacios de $\C^4$ que sean $T$-invariantes y tales que su suma directa sea $\C^4$.
            \begin{mdframed}[style=s]
                
            \end{mdframed}
    \end{enumerate}
        \item Sea $T\in L(\R^2)$ tal que \[[T]_\E=A=\begin{pmatrix}
        (2&1\\0&2)
    \end{pmatrix}\]
    \begin{enumerate}
        \item Probar que $W_1=\overline{\{(1,0)\}}$ es $T$-invariante.
            \begin{mdframed}[style=s]
                
            \end{mdframed}
        \item Probar que no existe un subespacio $W_2$ de $\R^2$ que sea $T$-invariante y que además $\R^2=W_1 \oplus W_2$.
            \begin{mdframed}[style=s]
                
            \end{mdframed}
    \end{enumerate}
        \item Sea \[A=\begin{pmatrix}
        1&-1\\2&2
    \end{pmatrix}\]
    \begin{enumerate}
        \item Si $T\in L(\R^2)$ es tal que $[T]_\E=A$ ¿Existe algún subespacio propio de $\R^2$ que sea $T$-invariante?
            \begin{mdframed}[style=s]
                
            \end{mdframed}
        \item Si $S\in L(\C^2)$ (pensado a $\C^2$ como $\C$-EV), es tal que $[S]_\E=A$ ¿Existe algún subespacio propio de $\C^2$ que sea $T$-invariante?
            \begin{mdframed}[style=s]
                
            \end{mdframed}
    \end{enumerate}
        \item Sean $V$ un $\K$-EV y $T,S\in L(V)$.
    \begin{enumerate}
        \item Sea $W$ un subespacio de $V$ que es invariante por $T$ y $S$. Probar que $W$ también es invariante por los operadores $T+S$ y $T\circ S$.
            \begin{mdframed}[style=s]
                
            \end{mdframed}
        \item Supongamos que $S\circ T=T\circ S$. Probar que si $\lambda$ es un autovalor de $T$, entonces el autoespacio asociado a $\lambda$ es $S$-invariante.
            \begin{mdframed}[style=s]
                
            \end{mdframed}
    \end{enumerate}
        \item ¿Se puede afirmar que un operador lineal tiene como único autovalor al cero, el nilpotente?
    \begin{mdframed}[style=s]
        
    \end{mdframed}
        \item Sea $N\in\C^{2\times2}$ tal que $N^2=0$. Probar que, o bien $N=0$, o bien $N$ es semejante a la matriz\[A=\begin{pmatrix}
        0&0\\1&0
    \end{pmatrix}\]
    \begin{mdframed}[style=s]
        
    \end{mdframed}
        \item Sea $V$ un $\K$-EV tal que $dim(V)=n$ y $B=\{b_1,\dots,b_n\}$ es una base (ordenada) de $V$. Sea $T\in L(V)$ dado por \[T(b_j)=\begin{cases}
        b_{j+1}&\quad\text{si }j=1,\dots,n-1\\
        0&\quad\text{si }j=n
    \end{cases}\]
    \begin{enumerate}
        \item Probar que el único autovalor de $T$ es cero.
            \begin{mdframed}[style=s]
                
            \end{mdframed}
        \item Probar que $T$ es nilpotente.
            \begin{mdframed}[style=s]
                
            \end{mdframed}
        \item ¿Cuánto vale la traza de $T$?¿Y el determinante?
            \begin{mdframed}[style=s]
                
            \end{mdframed}
    \end{enumerate}
        \item Sea $V$ un $\K$-EV tal que $dim(V)=n$. Probar que si $N\in L(V)$ es nilpotente, entonces $p_N(x)=x^n$
    \begin{mdframed}[style=s]
        
    \end{mdframed}
    \end{enumerate}
\end{document}